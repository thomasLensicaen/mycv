\documentclass[6pt]{article}

\usepackage[margin=0.5cm]{geometry}
\usepackage{array, xcolor}
\usepackage{enumitem}
\setlist{nosep}
\definecolor{lightgray}{gray}{0.8}
\newcolumntype{L}{>{\raggedleft}p{0.14\textwidth}}
\newcolumntype{R}{p{0.8\textwidth}}
\newcommand\VRule{\color{lightgray}\vrule width 0.5pt}
\newcommand\HRule{\color{lightgray}\hrule width 0.5pt}


\usepackage{tikz} % Required for creating the plots
\usetikzlibrary{shapes, backgrounds}
\tikzset{x=1cm, y=1cm} % Default tikz units

% Command to vertically centre adjacent content
\newcommand{\vcenteredhbox}[1]{% The only parameter is for the content to centre
	\begingroup%
		\setbox0=\hbox{#1}\parbox{\wd0}{\box0}%
	\endgroup%
}

\newcounter{barcount}

% Environment to hold a new bar chart
\newenvironment{barchart}[1]{ % The only parameter is the maximum bar width, in cm
	\newcommand{\barwidth}{0.35}
	\newcommand{\barsep}{0.2}
	
	% Command to add a bar to the bar chart
	\newcommand{\baritem}[2]{ % The first argument is the bar label and the second is the percentage the current bar should take up of the total width
		\pgfmathparse{##2}
		\let\perc\pgfmathresult
		
		\pgfmathparse{#1}
		\let\barsize\pgfmathresult
		
		\pgfmathparse{\barsize*##2/100}
		\let\barone\pgfmathresult
		
		\pgfmathparse{(\barwidth*\thebarcount)+(\barsep*\thebarcount)}
		\let\barx\pgfmathresult
		
		\filldraw[fill=black, draw=none] (0,-\barx) rectangle (\barone,-\barx-\barwidth);
		
		\node [label=180:\colorbox{black}{\textcolor{white}{##1}}] at (0,-\barx-0.175) {};
		\addtocounter{barcount}{1}
	}
	\begin{tikzpicture}
		\setcounter{barcount}{0}
}{
	\end{tikzpicture}
}


\title{\vspace{-3ex}\bfseries\Huge Thomas Louis - Data scientist\vspace{-3ex}}
\date{}
%\author{thomas.louis.pro@gmail.com}

\begin{document}
\begingroup
\let\center\flushleft
\let\endcenter\endflushleft
\maketitle
\endgroup
\begin{minipage}[ht]{0.48\textwidth}
201 Rue de Vaugirard\\
75015 Paris
\end{minipage}
\begin{minipage}[ht]{0.48\textwidth}
thomas.louis.pro@gmail.com
\end{minipage}
\section*{Experience Professionnelle}
\begin{tabular}{L!{\VRule}R}
06-2018--?&\textbf{Data Scientist, Finastra, Paris}\\
	&Finastra est un leader mondial de solutions logicielles pour la finance. Je travaille au sein du departement Trade \& Capital Market dans l'equipe d'innovation sur differents projets visant \`{a} ajouter des fonctionnalit\'{e}s bas\'{e}es sur l'intelligence artificielle dans les logiciels de trading
	\begin{itemize}
		\item Detection d'erreurs, et d'annomalies dans les trades en temps r\'{e}el
		\item Recherche et developement de nouveaux algorithmes de classification, et de model explainer pour le projet 'Fusion Detect'
		\item Recherche et developement de nouvelles m\'{e}thode d'optimization d'hyperparam\'{e}tres pour le projet 'Fusion Detect'
		\item Recherche et developpement de methodes d'analyses de series temporelles financi\`{e}re bas\'{e} sur le Deep Learning pour le projet 'Fusion Detect'
		\item Recherche et developpement de modelisation de march\'{e} bas\'{e} sur le Deep Learning pour un projet d'innovation
		\item J'ai contribu\'{e} en grande partie \`{a} un projet interne innovant de generation artificielle de donn\'{e}e bas\'{e} sur des techniques de Deep Learning  
	\end{itemize}
\\[5pt]
09-2015--06-2018&\textbf{Consultant Data Scientist, Amadeus, Sophia Antipolis}\\
&Amadeus est un leader mondial de solutions logicielles dans l'industrie du voyage. Elle offre entre autre un moteur de recherche de vols d'avion.
\begin{itemize}
	\item Modelisation predictive du choix des utilisateurs pour leurs choix de voyages en avion sur le set de recommandations, et du meilleur moment pour reserver un vol d'avion
%		\begin{itemize}
%			\item J'ai participer a un projet d'innovation au sein du l'incubateur d'Amadeus au sein d'une equipe creer pour. L'objectif etait de comprendre le choix et le comportement utilisateur. 3 delivrables ont ete creer:\\
%			\begin{enumerate}
%			\item Un estimateur du meilleur moment ou acheter un billet d'avion bas'e sur un dataset anonymis'e de reservation (voyage d'entreprise)
%			\item Un model de choix d'utilisateur de reservation de vol par rapport a un ensemble fixe de recommandation de vol.
%			\item Une liste de nouvelles idees qui sont susceptible de mener a de nouvelles opportunit'es business pour Amadeus
%			\end{enumerate}
%		\item Creation d'un dataset d'apprentissage en reliant les reservations aux recherches fait sur le moteur d'Amadeus
%		\item Creation d'un data pipeline big data sur spark puis les modeles predictifs en utilisant spark-mlib
%		\end{itemize}
	\item D\'{e}veloppement d'application Big Data et de data product
		\begin{itemize}
		\item Le but est de creer des outils de BI avanc'es capables de travailler sur des plusieurs teraoctets afin de permettres aux agences aeriennes et aux agences de voyage d'obtenir les nouvelles tendances des demandes de voyages.
		\item J'ai developp\'{e} et benchmark\'{e} le pipeline de data en spark et utilis\'{e} elasticsearch comme moteur de recherche
		\item J'ai utilis\'{e} le Data Mining/Fouille de donn\'{e}ees afin de detecter le trafic robotique et de l'eliminer
		\item J'ai utilis\'{e} des techniques de sampling avanc\'{e}es afin de reduire le temps de reponse
		\end{itemize}
	\item Analyse statistique \`{a} la demande selon les projets sur petits et grands datasets (jusqu'\`{a} 10 To)
\end{itemize}

\\[5pt]
03-2015--08-2015&\textbf{Stagiaire Data Scientist, Isoft, Ile de France}\\
&Isoft est une entreprise qui fournit des logiciels d'analyse de donn\'{e}e et de recherche op\'{e}rationnelle, c'est aussi un leader sur le march\'{e} fran\c{c}ais de la detection de fraude bancaire sur les paiements carte. Mon stage s'intitulait 'Detection de fraude bancaire et Clustering de graphe'.
\begin{itemize}
	\item Lecture d'article et creation d'un \'{e}tat de l'art des techniques de clustering de graphes et \'{e}tude des applications 
	\item Design et implementation d'algorithmes bas\'{e}s sur la th\'{e}orie des graphes (clustering, generation aleatoire de graphe, projection de graphe, ...) 
	\item Development d'un pipeline machine learning utilisant ces outils afin d'ameliorer la detection de fraude.
	\item Etudes statistiques des resultats obtenus sur des donn\'{e}es r\'{e}elles mais anonymis\'{e}es.
\end{itemize}
\end{tabular}
\\
\begin{tabular}{L!{\VRule}R}
\\[5pt]
10-2014--03-2015&\textbf{Projet de fin d'etude, Morpho (Safran)}\\
&Etude sur la securite d'Android Lollipop
\begin{itemize}
	\item Lecture et Recherche sur les technologies et vulnerabilit\'{e}s de l'OS 
	\item Realisation d'un proof of concept d'une attaque sur l'OS 
	\item Etude du Trusted Execution Environment d'Android
	\item Benchmarking de l'algorithme de reconnaissance faciale d'Android 
\end{itemize}

\\[5pt]
05-2014--08-2014&\textbf{Stagiaire Ingenieur Recherche, Biometry, Lucerne - Suisse}\\
&Reconnaissance faciale sur smartphone pour application bancaire
\begin{itemize}
	\item Lecture de l'etat de l'art des techniques de reconnaissance faciale 
	\item Implementation d'algorithme de reconnaissance faciale sur Android 
	\item Proof of Concept de nouvelles techniques d'authentification cloud. 
\end{itemize}
\end{tabular}

\section*{Education}
\begin{tabular}{L!{\VRule}R}
2012--2015&{\bf Dipl\^{o}me d'ingenieur mention Informatique, ENSICAEN.}\\[5pt]
2013--2015&{\bf Master Recherche en S\'{e}curite Informatique}, Universit\'{e} de Basse-Normandie.\\[5pt]
2016&Ceritification Machine Learning, Coursera/Stanford\\[5pt]
2017&Ceritification Deep Learning, Coursera/Stanford\\
\end{tabular}

\section*{Technologies et Comp\'{e}tences}
\hfill % Whitespace between
\begin{minipage}[t]{0.9\textwidth} % 10% of the page for the skills bar chart
	\vspace{-\baselineskip} % Required for vertically aligning minipages
	\begin{barchart}{10.5}
		\baritem{Python}{95}
		\baritem{Machine Learning}{90}
		\baritem{Deep Learning}{90}
		\baritem{Statistiques}{80}
		\baritem{Mathematiques Appliqu\'{e}es}{80}
		\baritem{Spark}{70}
		\baritem{Elasticsearch}{65}
		\baritem{Docker}{30}
		\baritem{C++}{15}
		\baritem{Rust}{10}
	\end{barchart}
\end{minipage}\\
\textbf{Autres Comp\'{e}tences:}\\
Git, Tensorflow/Keras, scikit-learn, Hive, Hadoop, Bash, Linux, Computer Vision, Biometrie, SQL 
\section*{Languages}
\begin{tabular}{L!{\VRule}R}
Fran\c{c}ais&Langue maternelle\\
Anglais&Courant\\
\end{tabular}

\section*{Autres}
\begin{itemize}
	\item Boxe Tha\"{i} / Krav Maga / Ski / Surf
	\item Permis de conduire
	\item PSC1
\end{itemize}



\end{document}
