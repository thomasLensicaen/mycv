\documentclass[6pt]{article}

\usepackage[margin=0.5cm]{geometry}
\usepackage{array, xcolor}
\usepackage{enumitem}
\setlist{nosep}
\definecolor{lightgray}{gray}{0.8}
\newcolumntype{L}{>{\raggedleft}p{0.14\textwidth}}
\newcolumntype{R}{p{0.8\textwidth}}
\newcommand\VRule{\color{lightgray}\vrule width 0.5pt}
\newcommand\HRule{\color{lightgray}\hrule width 0.5pt}


\usepackage{tikz} % Required for creating the plots
\usetikzlibrary{shapes, backgrounds}
\tikzset{x=1cm, y=1cm} % Default tikz units

% Command to vertically centre adjacent content
\newcommand{\vcenteredhbox}[1]{% The only parameter is for the content to centre
	\begingroup%
		\setbox0=\hbox{#1}\parbox{\wd0}{\box0}%
	\endgroup%
}

\newcounter{barcount}

% Environment to hold a new bar chart
\newenvironment{barchart}[1]{ % The only parameter is the maximum bar width, in cm
	\newcommand{\barwidth}{0.35}
	\newcommand{\barsep}{0.2}
	
	% Command to add a bar to the bar chart
	\newcommand{\baritem}[2]{ % The first argument is the bar label and the second is the percentage the current bar should take up of the total width
		\pgfmathparse{##2}
		\let\perc\pgfmathresult
		
		\pgfmathparse{#1}
		\let\barsize\pgfmathresult
		
		\pgfmathparse{\barsize*##2/100}
		\let\barone\pgfmathresult
		
		\pgfmathparse{(\barwidth*\thebarcount)+(\barsep*\thebarcount)}
		\let\barx\pgfmathresult
		
		\filldraw[fill=black, draw=none] (0,-\barx) rectangle (\barone,-\barx-\barwidth);
		
		\node [label=180:\colorbox{black}{\textcolor{white}{##1}}] at (0,-\barx-0.175) {};
		\addtocounter{barcount}{1}
	}
	\begin{tikzpicture}
		\setcounter{barcount}{0}
}{
	\end{tikzpicture}
}


\title{\vspace{-3ex}\bfseries\Huge Thomas Louis - Data scientist\vspace{-3ex}}
\date{}
%\author{thomas.louis.pro@gmail.com}

\begin{document}
\begingroup
\let\center\flushleft
\let\endcenter\endflushleft
\maketitle
\endgroup
\begin{minipage}[ht]{0.48\textwidth}
+336.22.84.85.15
\end{minipage}
\begin{minipage}[ht]{0.48\textwidth}
thomas.louis.pro@gmail.com
\end{minipage}
\section*{Professional Experience}
\begin{tabular}{L!{\VRule}R}
06-2018--present&\textbf{Data Scientist, Finastra, Paris}\\
	&Finastra is a world class financial software editor leader. I am working inside Trade \& Capital Market department in the innovation team on several projects aiming at adding artificial intelligence based functionalities into trading softwares
	\begin{itemize}
		\item I was manly in charge of developping, maintaining and improving Fusion Detect product: 
			\begin{itemize}
		              \item R\&D on classification problem and model explaination for Fusion Detect product
		              \item R\&D on hyperparameter optimization for Fusion Detect product 
		              \item R\&D on Timeseries Analysis based on Deep learning and statistical methods for Fusion Detect product
		              \item Benchmarking our algorithms on new datasets 
			\end{itemize}
		\item I have worked on an innovation project which aims at modeling market by solving stochastic differential equations with Deep learning
		\item I contributed in an internal hackathon to develop a generative model of artificial data
	\end{itemize}
\\[5pt]
09-2015--06-2018&\textbf{Data Scientist Consultant, Amadeus, Sophia Antipolis}\\
&Amadeus is the world leader software solution provider for the travel industry. It provides among other things a search engine for flight recommandation 
\begin{itemize}
	\item I have developed a predictive model of user choice among a set of recommandation for business trips in an innovation project.
	\item I have studied the companies' loss due to employees late booking timing.
	\item I have developed big data applications and data product
		\begin{itemize}
			\item the goal is to create advanced BI tools able to work on terabytes to allow travel agencies and airlines to get information about new tendencies for the flight demand.
			\item I have developped and benchmarked the data pipeline with spark and used elasticsearch as a search engine
			\item I used data mining technics to detect robotic/artificial trafic
			\item I used advanced sampling technics in order to reduce the response time while keeping a relevant response quality
		\end{itemize}
	\item I have made on demand statistical analysis for different projects on small and big datasets (up to 10 To)
\end{itemize}

\\[5pt]
03-2015--08-2015&\textbf{Data Scientist Intern, Isoft, Ile de France}\\
&Isoft is software company that develop operational research and statistical software. It is also a leader on the french market for fraud detection on e-banking. My internship was intitled 'Fraud detection and graph clustering'. 
\begin{itemize}
	\item I have read litterature on graph clustering technics and application. I have written a state of the art on this subject.
	\item I have designed and implemented graph theory based algorithms (clustering, artificial graph generation, bi-graph projection)
	\item I have developed and benchmarked a machine learning pipeline using graph clustering and classification technics
	\item I have made a statistical analysis over the prediction of this machine learning pipeline on real anonymized data
\end{itemize}
\end{tabular}
\\[5pt]
\begin{tabular}{L!{\VRule}R}
%\\[5pt]
%10-2014--03-2015&\textbf{Final year project/Master thesis, Morpho (Safran)}\\
%&Study about Android Lollipop security
%\begin{itemize}
%	\item I have read litterature about OS'specific vulnerabilities 
%	\item I have implemented a proof of concept of one of those attack 
%	\item I have written a study about the Trusted Execution Environment 
%	\item I have benchmarked the facial recognition algorithm of Android 
%\end{itemize}
%\\
\\[5pt]
05-2014--08-2014&\textbf{Research Engineer Intern, Biometry, Lucerne - Suisse}\\
&Facial Recognition on smartphone for e-banking application
\begin{itemize}
	\item I have read the state of the art of facial recognition technics
	\item I implemented facial recognition algorithm on Android
	\item I have implemented a proof of concept of strong authentication on cloud
\end{itemize}
\end{tabular}

\section*{Education}
\begin{tabular}{L!{\VRule}R}
2012--2015&{\bf Engineer Diploma, ENSICAEN.}\\[5pt]
2013--2015&{\bf MsC in Cyber Security}, Universit\'{e} de Basse-Normandie.\\[5pt]
2016&Machine Learning Ceritification, Coursera/Stanford\\[5pt]
2017&Deep Learning Ceritification, Coursera/Stanford\\
\end{tabular}

\section*{Technologies and Skills}
\hfill % Whitespace between
\begin{minipage}[t]{0.9\textwidth} % 10% of the page for the skills bar chart
	\vspace{-\baselineskip} % Required for vertically aligning minipages
	\begin{barchart}{10.5}
		\baritem{Python}{95}
		\baritem{Machine Learning}{90}
		\baritem{Deep Learning}{90}
		\baritem{Statistics}{80}
		\baritem{Applied Mathematics}{80}
		\baritem{Spark}{70}
		\baritem{Elasticsearch}{65}
		\baritem{Docker}{30}
		\baritem{C++}{15}
		\baritem{Rust}{10}
	\end{barchart}
\end{minipage}\\
\textbf{Other Skills:}\\
Git, Tensorflow/Keras, scikit-learn, Hive, Hadoop, Bash, Linux, Computer Vision, Biometrics, SQL 
\section*{Languages}
\begin{tabular}{L!{\VRule}R}
French&Native language\\
English&Fluent\\
\end{tabular}

\section*{Miscellaneous}
\begin{itemize}
	\item Thai Boxing / Ski / Surf
	\item Driving License 
	\item First Aid Diploma 
\end{itemize}



\end{document}
